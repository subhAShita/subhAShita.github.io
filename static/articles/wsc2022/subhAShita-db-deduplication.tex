% Template - https://sanskrit.uohyd.ac.in/18WSC/Style_files/CS_and_DH.tex


\documentclass[11pt]{article}
\usepackage{scl}
\usepackage{times}
\usepackage{url}
\usepackage{latexsym}
\usepackage{lineno}

\usepackage{fontspec, xunicode, xltxtra}
\newfontfamily\skt[Script=Devanagari]{Sanskrit 2003}

\title{A universal subhāṣita database, and deduplicating sanskrit texts}

\author{Vishvas Vasuki \\
  Dyugaṅgā, Beṅgaḷūru - 92 \\
  {\tt vishvas.vasuki+ACAD@gmail}
\\}

\date{}

\begin{document}
\maketitle
%\linenumbers
\begin{abstract}
Subhāṣita-s are popular and beautiful Sanskrit quotations. In this paper, we propose a universal open-source subhāṣita database. As a part of it, we present an algorithm for deduplicating sanskrit texts, as well as one for identifying non-duplicate vairants of a given quote.
\end{abstract}

\section{Motivation}
Subhāṣitas are popular and beautiful Sanskrit quotations - they're usually verses. One of the greatest (and useful) pleasures I've had in tough times is retreat for a while into the world of beautiful Subhāṣitas- so as to burst forth with renewed wisdom and energy. 

Since ages, they have been lovingly compiled \cite{subhashita-1952} and memorized. I especially like online collections curated by some friends and myself since a book is not always available, and I want to collect + easily access choice ones for future enjoyment. But it is tedious (atleast for me) to sit in front of a computer to do the following:

\begin{itemize}

\item
  read them,
\item
  or scour the internet for new ones
\item
  or collect favorites in a spreadsheet
\item
  or just annotate them with comments.
\end{itemize}

So, it is desirable to make the above as simple and easy as possible, and to share our collective labor so that we can benefit more easily from each others' work.

\section{A universal database}

We've set out to build a database of Subhāṣitas - which is:

\begin{itemize}

\item
  \textbf{Universal}

  \begin{itemize}
  
  \item
    Its goal is to contain within it every worthy subhAShita*- ever
    composed.
  \item
    In fact, the ambition encompasses all \emph{languages}, \emph{verse-
    and }prose- forms.
  \end{itemize}
\item
  \textbf{Freely and easily available}. Anyone should be able to

  \begin{itemize}
  
  \item
    Access it
  \item
    Export it to other formats
  \item
    Present it in any way users will find convenient.
  \end{itemize}
\item
  \textbf{Growing constantly in number}, thanks to contemporary compositions.
\item
  \textbf{Growing constantly in annotations/ ratings}, where annotations include rating, description, translations, metre, flaws,
    sources \ldots.
\end{itemize}

\section{Expected extensions}
We hope that this will motivate other such long-sought-after universal databases for sanskrit connoiseurs, like: one for metres.

The front-end clients built for this database could serve as a model for other kAvya readers. Similarly, one can build a collaboratively annotated and rated collection of verses/ sentences within the context of long sequential works (rather than free floating subhAShita-s).

% include your own bib file like this:
\bibliographystyle{acl}
\bibliography{scl}
\end{document}